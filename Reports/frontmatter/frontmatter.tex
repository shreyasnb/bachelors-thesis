% we include the glossary here (frontmatter is included with \input, so this command is as if it was in main.tex)
%All acronyms must be written in this file.
\newacronym{RTS}{RTS}{Real-Time System}
\newacronym{GPOS}{GPOS}{General Purpose Operating System}
\newacronym{RTOS}{RTOS}{Real-Time Operating System}
\newacronym{PGF}{PGF}{Portable Graphics Format}


\frontmatter % Use roman page numbering style (i, ii, iii, iv...) for the pre-content pages

\pagestyle{plain} % Default to the plain heading style until the thesis style is called for the body content

%----------------------------------------------------------------------------------------
%	TITLE PAGE
%----------------------------------------------------------------------------------------

\maketitlepage


%----------------------------------------------------------------------------------------
%	STATEMENT of INTEGRITY
%----------------------------------------------------------------------------------------
\integritystatement%

%----------------------------------------------------------------------------------------
%	DEDICATION  (optional)
%----------------------------------------------------------------------------------------
%
%\dedicatory{For/Dedicated to/To my\ldots}
% \begin{dedicatory}
% The dedicatory is optional. Below is an example of a humorous dedication.

% "To my wife Marganit and my children Ella Rose and Daniel Adam without whom this book would have been completed two years earlier." in "An Introduction To Algebraic Topology" by Joseph J. Rotman.
% \end{dedicatory}

%----------------------------------------------------------------------------------------
%	ABSTRACT PAGE
%----------------------------------------------------------------------------------------

\begin{abstract}

% here you put the abstract in the main language of the work.

Mechanical systems are most often described by a set of continuous-time, nonlinear, second-order differential equations (SODEs) of a particular structure governed by the covariant derivative. The digital implementation of controllers for such systems requires a discrete model of the system and hence requires numerical discretization schemes. Feedback linearizability of such sampled systems, however, depends on the discretization scheme employed. \\\\ In this thesis, we utilize retraction maps and their lifts to construct feedback linearizable discretizations for SODEs which can be applied to many mechanical systems.

\end{abstract}


%----------------------------------------------------------------------------------------
%	ACKNOWLEDGEMENTS (optional)
%----------------------------------------------------------------------------------------

\begin{acknowledgements}

I would like to thank Prof.~David M. Diego, Insituto de Ciencias Matemáticas (ICMAT), for his guidance in the development of the theory in this document.

I would also like to acknowledge the support of my family, friends, and colleagues, who have supported me throughout the development of this work.

\end{acknowledgements}

%----------------------------------------------------------------------------------------
%	LIST OF CONTENTS/FIGURES/TABLES PAGES
%----------------------------------------------------------------------------------------

\tableofcontents % Prints the main table of contents

\listoffigures % Prints the list of figures

\listoftables % Prints the list of tables

% \listofalgorithms % Prints the list of algorithms
% \addchaptertocentry{\listalgorithmname}


\renewcommand{\lstlistlistingname}{List of Source Code}
% \lstlistoflistings % Prints the list of listings (programming language source code)
% \addchaptertocentry{\lstlistlistingname}


%----------------------------------------------------------------------------------------
%	ABBREVIATIONS
%----------------------------------------------------------------------------------------
%\begin{abbreviations}{ll} % Include a list of abbreviations (a table of two columns)
%%\textbf{LAH} & \textbf{L}ist \textbf{A}bbreviations \textbf{H}ere\\
%%\textbf{WSF} & \textbf{W}hat (it) \textbf{S}tands \textbf{F}or\\
%\end{abbreviations}

%----------------------------------------------------------------------------------------
%	SYMBOLS
%----------------------------------------------------------------------------------------

\begin{symbols}{ll} % Include a list of Symbols (a two column table)

$M$ & manifold \\
$\gamma$ & curve/geodesic on $M$ \\
$TM$ & tangent bundle of $M$\\
$\mathfrak{X}(M)$ & space of vector fields on $M$\\
$X$ & vector field on $M$\\
$TTM$ & double tangent bundle of $M$\\
$\varphi$ & diffeomorphism \\
$T \varphi$ & tangent lift of $\varphi$ \\
$\mathcal{R}$ & retraction map \\
$\mathcal{D}$ & discretization map \\
$\tau_M$ & canonical projection map from $TM$ \\
$\kappa_M$ & canonical involution map on $TTM$ \\
$f \circ g$ & composition of $f$ and $g$ \\
$\mathbb{I}$ & identity map \\
$n$ & default dimension of $M$ \\
$\langle \cdot, \cdot \rangle$ & inner product \\
$\partial_x$ & partial derivative with respect to $x$ \\
$h$ & step-size of discretization \\
%Symbol & Name & Unit \\

\end{symbols}



%----------------------------------------------------------------------------------------
%	ACRONYMS
%----------------------------------------------------------------------------------------

\newcommand{\listacronymname}{List of Acronyms}

%Use GLS
\glsresetall
\printglossary[title=\listacronymname,type=\acronymtype,style=long]

%----------------------------------------------------------------------------------------
%	DONE
%----------------------------------------------------------------------------------------

\mainmatter % Begin numeric (1,2,3...) page numbering
\pagestyle{thesis} % Return the page headers back to the "thesis" style
