\chapter{Feedback Linearization}

\section{Introduction}

In this chapter, we will understand the concept of feedback linearization, and its application to mechanical systems, which gives rise to an interesting class of \textbf{Mechanically Feedback Linearizable} systems.

\section{Feedback Linearization}

Feedback linearization is a control technique used to transform a nonlinear system into an equivalent linear system through a change of variables and a suitable feedback control law. This method allows the application of linear control techniques to nonlinear systems, which can simplify the design and analysis of control systems.

Consider the following continuous-time dynamical system (for $t \in [0, T ], T > 0$):

\begin{equation}
    \label{eq:cts}
    \dfrac{d}{dt}x(t) = X(x(t), u(t))
\end{equation}

on an $n$-dimensional manifold $M$, where $X(\cdot, u) \in \mathfrak{X}(M)$ is a vector field, for each $u \in U \subset \R[n]$. A point $(x_0, u_0) \in M \times U$ is called an equilibrium point of the system~\eqref{eq:cts} if $X(x_0, u_0) = 0$.

\begin{defn}
    Let $M$ and $N$ be two $n$-dimensional manifolds and $\varphi: M \lra N$ be a diffeomorphism. Let $X \in \mathfrak{X}(M)$ be a vector field on $M$. Then, $X_{\varphi} = T\varphi \circ X \circ \varphi^{-1}$ is a vector field on $N$ (push-forward) for the dynamical system
    \begin{equation}
        \dfrac{d}{dt}\tilde{x}(t) = X_{\varphi}(\tilde{x}(t), u(t))
    \end{equation}
    with $\tilde{x}(0) = \varphi(x(0))$ satisfying $\tilde{x}(t) = \varphi(x(t)), t \in [0,T]$.

    Let $x \in \mathcal{O}(x_0)$ and $u \in \mathcal{O}(u_0)$ be open balls (neighborhood) around $x_0$ and $u_0$ in $M$ and $U$ respectively. Let $x \mapsto \varphi(x) = \tilde{x} \in N : = \R[n]$ be a diffeomorphism, and $(x,u) \mapsto \psi(x,u) : = v \in \R[m]$ such that for each fixed $x$, $\psi(x, \cdot) : U \lra \R[n]$ is invertible. Thus, a dynamical system \eqref{eq:cts} is said to be (locally) feedback linearizable around $(x_0, u_0)$ on $\mathcal{O}(x_0) \times \mathcal{O}(u_0)$ if there exists matrices $A \in \R[n \times n]$ and $B \in \R[n \times m]$ such that $X_{\varphi}(\tilde{x}, v) = A \tilde{x} + Bv$, with $v = \psi(\varphi^{-1}(\tilde{x}), u)$. The feedback linearized dynamical system is given by:

    \begin{equation}
        \dfrac{d}{dt}\tilde{x}(t) = A \tilde{x}(t) + B v(t)
    \end{equation}
\end{defn}



\section{Mechanical Feedback Linearization}
Mechanical Feedback Linearization (MF-Linearization) is given defined as:

\begin{itemize}
    \item change of coordinates given by the diffeomorphism
    \begin{equation}
        \begin{split}
            \varPhi: & TM \lra T\widetilde{M} \\
            & (x,v) \mapsto (\tilde{x}, \tilde{v}) = (\varphi(x), D \varphi(x)v)
        \end{split}
    \end{equation}
    \item mechanical feedback transformations, denoted $(\alpha, \beta, \gamma)$, of the form
    \begin{equation}
        u_r = \gamma^r_{jk}(x) y^j y^k + \alpha^r (x) + \sum_{s=1}^m \beta^r_s(x) \tilde{u}_s
    \end{equation}
    where $\gamma^r_{jk} = \gamma^r_{kj}$
\end{itemize}
such that the transformed system is linear and mechanical:
\begin{equation}
    \begin{split}
        \dot{\tilde{x}}^i & = \tilde{v}^i \\
        \dot{\tilde{v}}^i & = E^i_j \tilde{x}^j + \sum_{s=1}^m b_s^i \tilde{u}_s
    \end{split} \tag{$\mathcal{LMS}$}
\end{equation}

Feedback linearization has been successfully applied to a wide range of nonlinear systems, including robotic manipulators, aerospace vehicles, and chemical processes. It provides a systematic approach to control design and can significantly improve the performance and stability of nonlinear systems.

In the following sections, we will explore the mathematical foundations of feedback linearization, discuss its application to various systems, and present examples to illustrate its effectiveness.