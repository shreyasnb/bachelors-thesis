% Chapter 2

\chapter{Mechanical Control Systems} % Main chapter title

\label{chap:sode} % For referencing the chapter elsewhere, use \ref{chap:Chapter2} 

Mechanical systems are usually described by nonlinear second-order differential equations (SODEs). In this chapter, we will discuss the geometric formulation of SODEs and their discretization. We will define different classes of mechanical systems, and how a specific class of mechanical systems can be controlled using a technique called \textit{feedback linearization}.

%----------------------------------------------------------------------------------------
\section{Second-order differential equations (SODEs)} 
Let $x \in M$ and $(x, \dot{x}) \in TM$ be the coordinates on the manifold $M$ and the induced coordinates on the tangent bundle of $M$, respectively. We know that a second-order differential equation is a vector field $X$ such that $\tau_{TM}(X) = T \tau_M (X)$. This implies that the vector field $X$ on $TM$ is a section of the second-order tangent bundle $TTM$. Locally, if we take coordinates $(x^i)$ on $M$ and induced coordinates $(x^i, \dot{x}^i)$ on $TM$, then:
\begin{equation}
    X = \dot{x}^i \frac{\partial}{\partial x^i} + X^i(x^i, \dot{x}^i) \frac{\partial}{\partial \dot{x}^i}
\end{equation}
To find the integral curves of $X$ is equivalent to solving the SODE:
\begin{equation}
\label{eq:sode}
    \dfrac{d^2}{dt^2}x(t) = X \left( x(t), \dfrac{d}{dt}x(t) \right) 
\end{equation}
Now, we wish to discretize this using the notion of the discretization map on $TM$. We would like to tangently lift a discretization on $M$ to obtain $\D^{TTM}: TTM \lra TM \times TM$ as defined in Proposition \ref{prop:lift-commutator}. This yields the following numerical scheme [\cite{21MBLDMdD}]:
\begin{equation}
\label{eq:disc}
\begin{split}
    h X \left( \left(\tau_{TM} \circ \left(\D^{TTM}\right)^{-1}\right)(x_k, y_k; x_{k+1}, y_{k+1})\right) \\ = \left(\D^{TTM}\right)^{-1} (x_k, y_k; x_{k+1}, y_{k+1})
\end{split}
\end{equation}

\subsection{Example}
Let us say we choose the midpoint  discretization on $N={\mathbb R}^n$, denoted by $\D$ of the following form:
\begin{equation}
    \D^{TN}(\tilde{x}, \tilde{y}) = \left(\tilde{x} - \dfrac{\tilde{y}}{2}, \tilde{x} + \dfrac{\tilde{y}}{2} \right)
\end{equation}
for some $(\tilde{x}, \tilde{y}) \in TN$.
Thus, similar to Example \ref{ex:midpoint-disc}, we have:

\begin{equation}
    \D^{TTN}(\tilde{x}, \dot{\tilde{x}}, \tilde{y}, \dot{\tilde{y}}) = \left( \tilde{x} - \dfrac{\tilde{y}}{2}, \tilde{x} + \dfrac{\tilde{y}}{2}, \dot{\tilde{x}} - \dfrac{\dot{\tilde{y}}}{2}, \dot{\tilde{x}} + \dfrac{\dot{\tilde{y}}}{2}\right)
\end{equation}
which is a discretization on $TN$.

Now, to lift $\D^{TTN}$ to obtain $\D^{TTM}$, we use Proposition \ref{prop:lift-commutator}, which gives:

\begin{equation}
    \D^{TTM} = (T \phi \times T \phi)^{-1} \circ \D^{TTN} \circ TT\phi
\end{equation}
 which is also a discretization map on $TM$.

 Using the numerical scheme from Equation \eqref{eq:disc}, we obtain:
 \begin{equation}
     \begin{split}
         \dfrac{x_{k+1} - x_k}{h} & = \dfrac{y_{k+1} + y_k}{2}, \\
         \dfrac{y_{k+1} - y_k}{h} & = X \left(\dfrac{x_k + x_{k+1}}{2}, \dfrac{y_k + y_{k+1}}{2} \right)
     \end{split}
 \end{equation}
 which is the numerical scheme for a symmetric discretization of the SODE \eqref{eq:sode}.

\section{Mechanical control systems}

We define a mechanical control system as proposed in [\cite{10076262}].

\begin{defn}
    A mechanical control system $(\mathcal{MS})_{(n,m)}$ is defined by a $4$-tuple $(M, \nabla, \mathfrak{g}, e)$ where:
    \begin{itemize}
        \item $M$ is an $n$-dimensional manifold
        \item $\nabla$ is a symmetric affine connection on $M$
        \item $\mathfrak{g} = \{g_1, \dots, g_m\}$ is an $m$-tuple of control vector fields on $M$
        \item $e$ is an uncontrolled vector field on $M$
    \end{itemize}
    $(\mathcal{MS})_{(n,m)}$ can be represented by the differential equation:
    \begin{equation}
        \label{eq:mech}
        \nabla_{\dot{x}} \dot{x} = e(x) + \sum_{r=1}^m g_r(x) u_r 
    \end{equation}
    Or equivalently in local coordinates $x = (x^1, \dots, x^n)$ on $M$, 
    \begin{equation}\label{SODE-initial}
        \ddot{x}^i = - \Gamma ^i_{jk}(x)\dot{x}^j \dot{x}^k + e^i(x) + \sum_{r=1}^m g^i_r(x)u_r
    \end{equation}
    where $\Gamma^i_{jk}$ are the Christoffel symbols corresponding to the Coriolis and centrifugal force terms, $e(x)$ is the uncontrolled vector field, $g_r(x)$ are the controlled vector fields in $Q$.

    If we write this as two first-order differential equations:
    \begin{equation}\label{SODE-nonlinear}
        \begin{split}
            \dot{x}^i  &= y^i; \\
            \dot{y}^i  &= - \Gamma^i_{jk}(x)y^jy^k + e^i(x) + \sum_{r=1}^m g_r^i(x)u_r
        \end{split} \tag{$\mathcal{MS}$} 
    \end{equation}
\end{defn}

\newpage

\subsection{Example}

Consider the classic example of an inverted pendulum on a cart:

\begin{figure}[h]
    \centering
    \begin{tikzpicture} [thick]

        % Angle of Pendulum
        
        % ground
        \draw [brown!80!red] (-2,0) -- (2,0);
        \fill [pattern = crosshatch dots,
            pattern color = brown!80!red] (-2,0) rectangle (2,-.2);
        
        % cart
        \begin{scope} [draw = orange,
            fill = orange!20, 
            dot/.style = {orange, radius = .025}]
        
        \filldraw [rotate around = {-30:(0,1.5)}] (.09,1.5) -- 
            node [midway, right] {$l$} 
            node [very near end, right] {$m$}
            +(0,2) arc (0:180:.09) 
            coordinate [pos = .5] (T) -- (-.09,1.5);
        
        \filldraw (-.65,.15) circle (.15);
        \fill [dot] (-.65,.15) circle;
        \filldraw (.65,.15) circle (.15);
        \fill [dot] (.65,.15) circle;
        
        \filldraw (-1,1.5) -- coordinate [pos = .5] (F)
            (-1,.3) -- node [right, above, near end] {$M$}
            (1,.3) -- (1,1.5) 
            coordinate (X) -- (.1,1.5)
            arc (0:180:.1) -- (-1.014,1.5);
        
        \fill [dot] (0,1.52) circle;
        \end{scope}
        
        % lines and angles
        \begin{scope} [thin, orange!50!black]
            \draw (T) -- (0,1.52) coordinate (P);
            \draw [dashed] (P) + (0,-2) -- +(0,2.2);
            \draw (P) + (0,.5) arc (90:90-30:.5) node [black, midway, above] {$\theta$};
            \draw [->] (0,-0.5) -- (1.2,-0.5) node [black, right] {$x$};
        \end{scope}

    \end{tikzpicture}
\end{figure}

The equations of motion for this system are given by:

\begin{equation}
    \begin{split}
        (M + m) \ddot{x} + ml \cos{\theta} \ddot{\theta} - m l \sin{\theta} {(\dot{\theta})}^2 = F \\
        ml \cos{\theta} \ddot{x} + \dfrac{4}{3} ml^2 \ddot{\theta} - mgl \sin{\theta} = 0
    \end{split}
\end{equation}

where $F$ is the force applied to the cart.

Let us define $x^1 = x, x^2 = \theta$ and we denote $\dot{x}^1 = y^1, \dot{x}^2 = y^2$. We can write the above equations as:

\begin{equation}
    \label{eq:mech-ex}
    \begin{split}
        \dot{x}^1 &= y^1 \\
        \dot{x}^2 &= y^2 \\
        \dot{y}^1 &= -\Gamma_{22}^1 y^2 y^2 + e^1 + g^1 u \\
        \dot{y}^2 &= -\Gamma_{22}^2 y^2 y^2 + e^2 + g^2 u
    \end{split}
\end{equation}

where, for $\eta = \dfrac{3}{ml^2 \left(4(M + m) - 3m \cos^2{\theta} \right)}$ we have:

\begin{equation*}
    \begin{split}
        \Gamma_{22}^1 &= \left(-\dfrac{4}{3} m^2 l^3 \sin^3{\theta}\right)\eta \hspace{50pt} \Gamma_{22}^2 = \left(\dfrac{1}{2} m^2 l^2 \sin{2\theta}\right) \eta \\
        e^1 &= \left(\dfrac{1}{2} m^2 l^2 g \sin{2\theta}\right) \eta \hspace{50pt} e^2 = \left( (M + m) mgl \sin{\theta} \right) \eta \\
        g^1 &= \left(\dfrac{4}{3} ml^2 \right) \eta \hspace{90pt} g^2 = \left( - ml \cos{\theta} \right) \eta 
    \end{split}
\end{equation*}

Thus, this system in \eqref{eq:mech-ex} is in the form of a mechanical control system \eqref{SODE-nonlinear}.

It is be interesting to note that this system is mechanically feedback linearizable only if the input is given to the pendulum (as torque) and not the cart!