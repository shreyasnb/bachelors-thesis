\documentclass{beamer}
%\documentclass[handout,t]{beamer}

\batchmode
% \usepackage{pgfpages}
% \pgfpagesuselayout{4 on 1}[letterpaper,landscape,border shrink=5mm]

\usepackage{amsmath,amssymb,enumerate,epsfig,bbm,calc,color,ifthen,capt-of}

\usetheme{Berlin}
\usecolortheme{mit}

\title{Feedback Linearizable Discretizations of Mechanical Systems using Retraction Maps} 
\author{Shreyas N B}
\date{\today}
\pgfdeclareimage[height=0.5cm]{iitb-logo}{../Figures/iitblogo.png}
\logo{\pgfuseimage{iitb-logo}\hspace*{0.3cm}}

\AtBeginSection[]
{
  \begin{frame}<beamer>
    \frametitle{Outline}
    \tableofcontents[currentsection]
  \end{frame}
}
\beamerdefaultoverlayspecification{<+->}
% -----------------------------------------------------------------------------
\begin{document}
% -----------------------------------------------------------------------------

\frame{\titlepage}

\section[Outline]{}
\begin{frame}{Outline}
  \tableofcontents
\end{frame}

% -----------------------------------------------------------------------------
\section{Introduction}
\subsection{Feedback Linearization}
\begin{frame}{Definitions}
  Let $M$ and $N$ be two $n$-dimensional manifolds and $\phi: M \to N$ be a diffeomorphism. Let $X \in \mathfrak{X}(M)$ be a vector field on $M$. Then, $X_{\phi} := T\phi \circ X \circ \phi^{-1}$ is a vector field on $N$.
  \begin{block}{Feedback Linearization}
    Let $x_0 \in \mathcal{O}(x_0)$ and $u_0 \in \mathcal{O}(u_0)$ be such that $f(x_0,u_0) = 0$. Then, the system is locally feedback linearizable if there exists a diffeomorphism $\phi: M \to N$ such that $X_{\phi} = \frac{\partial}{\partial x_n}$. 

  \end{block}

\end{frame}
\begin{frame}{MIT Hack}
  The HACK:
  \pause
  \begin{itemize}
    \item<2-> Tom O'Connor - wanted to measure Harvard bridge to track his progress when walking
    \item<3-> Smoot's height chosen as unit of measurement (he was the shortest)
    \item<4-> Seven freshman measured Harvard bridge using Smoot's body to mark distance
    \item<5-> Result: Harvard Bridge = 364.4 smoots (+an ear)
  \end{itemize}
\end{frame}
\begin{frame}{Career}
  \begin{itemize}
    \item<1-> Chairman of the American National Standards Institute 
    \item<2-> Served as president of the International Organization for Standardization from 2003 to 2005.
  \end{itemize}
\end{frame}
% 

\subsection{Retraction Maps}

\begin{frame}{What is a Smoot?}
  \begin{itemize}
    \item An imprecise unit of measurement originating from famous MIT hack
    \item 1 smoot = 5 feet and 7 inches
    \item Harvard bridge = 364.4 smoots (+ an ear)
  \end{itemize}
\end{frame}

\subsection{Mechanical Systems}

\begin{frame}{MIT Hack}
  The HACK:
  \pause
  \begin{itemize}
    \item<2-> Tom O'Connor - wanted to measure Harvard bridge to track his progress when walking
    \item<3-> Smoot's height chosen as unit of measurement (he was the shortest)
    \item<4-> Seven freshman measured Harvard bridge using Smoot's body to mark distance
    \item<5-> Result: Harvard Bridge = 364.4 smoots (+an ear)
  \end{itemize}
\end{frame}

\section{MF-Linearization}
\subsection{MF-Linearizability}
\begin{frame}{Career}
  \begin{itemize}
    \item<1-> Chairman of the American National Standards Institute 
    \item<2-> Served as president of the International Organization for Standardization from 2003 to 2005.
  \end{itemize}
\end{frame}

\subsection{Examples}
\begin{frame}{Examples}
  \begin{itemize}
    \item<1-> Inverted Pendulum
    \item<2-> Double Pendulum
    \item<3-> Cart-Pole System
  \end{itemize}
\end{frame}

\section{Conclusions}
\subsection{Some results}
\begin{frame}{Questions and Answers}
  Want to know more?

  \begin{itemize}
    \item Browse \url{http://web.mit.edu/smoot/history.htm}.
    \item Smoot's Legacy \url{http://alum.mit.edu/news/AlumniNews/Archive/smoots_legacy}.
    \item Smoot Salute! \url{http://web.mit.edu/spotlight/smoot-salute}.
  \end{itemize}

\end{frame}

\end{document}